% !TEX root = KezdoOldalak.tex
\chapter{Általam ismert és alkalmazni kívánt technológiák}\label{ch:ALTALAMISMERT}

\lstdefinelanguage{JavaScript}{
  sensitive=true,
    keywords={%
    % JavaScript
    typeof, new, true, false, catch, function, return, null, catch, switch, var, if, in, while, do, else, case, break,
    % HTML
    html, title, meta, style, head, body, script, canvas,
    % CSS
    border:, transform:, -moz-transform:, transition-duration:, transition-property:,
    transition-timing-function:
    },
    % http://texblog.org/tag/otherkeywords/
    otherkeywords={<, >, \/},   
    ndkeywords={class, export, boolean, throw, implements, import, this},   
    comment=[l]{//},
    % morecomment=[s][keywordstyle]{<}{>},  
    morecomment=[s]{/*}{*/},
    morecomment=[s]{<!}{>},
    morestring=[b]',
    morestring=[b]",    
    alsoletter={-},
    alsodigit={:}
}

\lstset{
   language=JavaScript,
   backgroundcolor=\color{lightgray},
   extendedchars=true,
   basicstyle=\footnotesize\ttfamily,
   showstringspaces=false,
   showspaces=false,
   numbers=left,
   numberstyle=\footnotesize,
   numbersep=9pt,
   tabsize=2,
   breaklines=true,
   showtabs=false,
   captionpos=b
}

\begin{osszefoglal}
Ennek a fejezetnek a célja, hogy a dolgozat belátást nyújtson az általam már ismert, kipróbált és alkalmazni kívánt technológiák történetébe, működésének rövid ismertetőjébe
\end{osszefoglal}

\section{React}\label{sec:ALTALAMISMERT:react}
\begin{figure}[h]
	\centering
	\includegraphics[scale=0.7]{React}
\end{figure}
\subsection{Miért is pont a React?}\label{subsec:ALTALAMISMERT:react:miertis}
Amint már említettem, a praktikám során találkoztam ezzel a keretrendszerrel, ott kellett alkossunk vele egy szabadságnyilvántartó webalkalmazást.

A mentorunk mondta, hogy jó lenne ezzel dolgozzunk, mivel egy új, gyorsan feltörekvő, egyre jobban terjedő javaScript alapú keretrendszer. 

Én a javaScriptet már addig is nagyon megszerettem könnyedsége miatt, bármit hihetetlenül egyszerűen meg leget vele valósítani, de a React még egyszerűbbé tette a dolgunk.
%%%%%%%%%%%%%%%%%%%%%%%%%%%%%%%%%%%%%%%%%%%%%%%%
\subsection{Mi is az a React?}\label{subsec:ALTALAMISMERT:react:miis}
A React.js egy javaScript állomány, melyet a Facebook mérnökei fejlesztettek ki. 2011-ben használták először a Facebook fejlesztői, majd 2013-ban nyílt forráskódúvá tették. 2015-ben megjelent a React Native, ezzel már Android-ra és iOS-re is könnyűszerrel lehet fejleszteni.

\newpage
NodeJS {\bf n}ode {\bf p}ackage {\bf m}anager-e (npm) segítségével tudjuk telepíteni, pontosabban a projektünk mellé letölteni a megfelelő csomagokat, amelyek elősegítik a React kódunk lefordítását, futtatását.
\begin{figure}[h]
	\centering
	\includegraphics[scale=0.2]{nodeJS}
\end{figure}

A React egy úgynevezett virtuális DOM-ot használ a módosítások végrehajtásához. Ha valami változás történik a weboldalon, megkeresi a DOM részfáját, és csak azt módosítja, így nem kell az egész weboldalnak frissülnie. Ezzel létrejöttek az újrafelhasználható komponensek, melyek nagyban megkönnyítik a User Interfacek (felhasználói felületek) létrehozásának módját.

A React JSX elemek használata segítségével még könnyebben átlátható és megérthető kódot alkothat.

Íme néhány ok, amiért érdemes megismerni:
\renewcommand{\baselinestretch}{0.98}\normalsize
\begin{description}
    \setlength{\itemsep}{0.04mm}
    \item[gyors] -- Alkalmazásai képesek kezelni a komplex változásokat, és még mindig gyorsaságot és megbízhatóságot mutatnak.
    \item[moduláris] -- Ahelyett, hogy nagy, hosszú kódokat írnánk, sok kisebb, újrafelhasználható fájlt írhatunk.
    \item[rugalmas] -- Olyan dolgokra is alkalmazható, aminek semmi köze a webapplikációkhoz
    \item[népszerű] -- Egyre jobban terjed, így ismeretével sok előnyünk lehet az informatika szakmában
\end{description}
%%%%%%%%%%%%%%%%%%%%%%%%%%%%%%%%%%%%%%%%%%%%%%%%%%%%
\subsection{JSX}\label{subsec:ALTALAMISMERT:react:jsx}
A JSX a JavaScript bővítése, azaz a böngésző nem képes olvasni, nem egy igazi JavaScritpt. Azért alkották meg, hogy a Reactal együtt használják.
Szükség van egy JSX fordítóra, amely hagyományos JavaScripté alakítja.

Egy JSX elem pont úgy néz ki mint egy HTML elem, de a JavaScript tulajdonságnak köszönhetően el is menthetem egy változóba,de akár objektumba is.

\subsubsection{Példák:}\label{subsubsec:ALTALAMISMERT:react:jsx:pl}
JSX elem elmentése egy változóba:
\lstinputlisting[caption=A JSX kód sokban hasonlít a HTML-re]{Codes/jsx.js}
JSX elem elmentése a \verb+myList+\ változóba és a változó kirenderelése a DOMra:
\lstinputlisting[caption=Az "app" Id-jű tag közé fog kerülni az elem]{Codes/jsxWithRender.js}
%%%%%%%%%%%%%%%%%%%%%%%%%%%%%%%%%%%%%%%%%%%%%%%%%%
\section{JSON}\label{sec:ALTALAMISMERT:json}
Már a webprogramozás tantárgy során be kellett lássam, a {\bf JavaScript Object Notation} hihetetlenül hasznos kis méretű szöveg alapú szabvány, mely kulcs-érték párok segítségével bármit eltud tárolni, természetesen szöveg formában.

Aztán a praktika során megmutatták, ezt a szabványt API-k milyen könnyedén lehet alkalmazni különböző technológiák közti információcserére. Nagy előnye a JSON objektumoknak az, hogy ennek az egyszerű adatstruktúrának több nyelvben is van értelmezője. Tehát objektumból JSON-t alkotva, a szöveg alapú objektumot átküldve más technológiáknak, majd ott JSON-ből objektumot visszaalakítva a kommunikáció játszi könnyedséggel létrejöhet.
\subsubsection{Példa:}\label{subsubsec:ALTALAMISMERT:react:json:pl}
\lstinputlisting[caption=JSON kód]{Codes/json.json}
%%%%%%%%%%%%%%%%%%%%%%%%%%%%%%%%%%%%%%%%%%%%%%%%%%%%%%%%%%
\section{MySQL-Wamp szerver}\label{sec:ALTALAMISMERT:mysql}
\begin{figure}[h]
	\centering
	\includegraphics[scale=0.35]{wamp}
\end{figure}
\subsection{A wamp virtuális szerver}\label{subsec:ALTALAMISMERT:mysql:virtuals}
A Wamp server lényegében egy webfejlesztő-keretrendszer Windows operációs rendszerhez. Tartalmazza a PHP, Apache, MySQL rendszereket egy csomagba összerakva, amely segítségével nagyon egyszerűen tudunk szervert létrehozni helyi gépen lokális tárolási lehetőséggel.

Ezzel a programmal a Webprogramozás tantárgy keretein belül találkoztam, megtetszett ez az összeépített csomag, így a lokális gépemen létrejött egy mySQL adatbázis, amit kedvem szerint tudtam feltölteni több különböző adatbázissal. A Wamp vizuális felülettel is rendelkezik, amely hihetetlenül megkönnyíti az adatbázis átláthatóságát.
%%%%%%%%%%%%%%%%%%%%%%%%%%%%%%%%%%%%%%%%%%%%%
\subsection{A mySQL-ről röviden}\label{subsec:ALTALAMISMERT:mysql:mysqlrolroviden}
A MySQL egy többfelhasználós, többszálú, SQL-alapú relációs adatbázis-kezelő szerver.
A szoftver eredeti fejlesztője a svéd MySQL AB cég volt, de 2008-ban a Sun felvásárolta, majd 2010-ben a Sun-t  az Oracle, így most a mySQL az Oracle birtokában van.

A MySQL az egyik legelterjedtebb adatbázis kezelő rendszer, teljesen nyílt forráskódú Linux–Apache–MySQL–PHP összeállítás, platformfüggetlen, számos programozási nyelv által elérhető, hatékony és egyszerűen beüzemelhető dinamikus webhelyek számára. 
%%%%%%%%%%%%%%%%%%%%%%%%%%%%%%%%%%%%%%%%%%%%%%