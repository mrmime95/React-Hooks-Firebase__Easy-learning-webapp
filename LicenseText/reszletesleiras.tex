\chapter{A project részletes leírása}\label{ch:RESZLETESLEIRAS}

\begin{osszefoglal}
Ebben a fejezetben az Online Napló részletesebb teve lesz olvasható.
\end{osszefoglal}

\section{A célom ismertetése}\label{sec:RESZLETESLEIRAS:cel}
Tehát ezen webalkalmazás ötlete a Canvas-szal való találkozásom eredménye, amikor elképzeltem mennyivel könnyebb lett volna minden a középiskolában ha egy ilyen alkalmazás áll a rendelkezésünkre.
%%%%%%%%%%%%%%%%%%%%%%%%%%%%%%%
\subsection{Felhasználó típusok}\label{subsec:RESZLETESLEIRAS:cel:felhasztip}
Az alkalmazásban 4 fajta felhasználó típust szeretnék megvalósítani: diák, diák szülei, tanár, titkár.

Mindezek mellé meg egy rendszerkezelő Admin is be kéne iktatódjon.
\subsubsection{Fontosabb jogok a különböző felhasználóknak:}\label{subsubsec:RESZLETESLEIRAS:cel:felhasztip:tablazat}
\begin{table}[h]
\begin{tabular}{|l|l|l|l|l|l|}
\hline
 & Admin & Diák & Tanár & Szülő & Titkár \\
\hline
Új felhasználó bevitele & \checkmark & & & & \\
\hline
Jegy adás & & & \checkmark & & \\
\hline
Üzenet küldés & \checkmark & \checkmark & \checkmark & \checkmark & \checkmark \\
\hline
Üzenet olvasás & \checkmark & \checkmark & \checkmark & \checkmark & \checkmark \\
\hline
Értesítés & \checkmark & & \checkmark & & \checkmark \\
\hline
Jegy aláírása & & & & \checkmark & \\
\hline
Osztály létrehozása & \checkmark & & & & \\
\hline
\end{tabular}
\end{table}
%%%%%%%%%%%%%%%%%%%%%%%%%%%%%%
\subsection{Admin}\label{subsec:RESZLETESLEIRAS:cel:admin}
Az {\bf Admin} felhasználónk az a felhasználó, aki a rendszer felhasználóival foglalkozik, azaz ő nem jegyet ad vagy hasonló dolgok.

Az ő feladata kifejezetten a felhasználókra irányul.

Ő az aki az iskola számára új diákot tud felvenni a rendszerbe, egy kezdetleges jelszóval ellátva, osztályt tud létrehozni, amelyekbe betehet egy adott tanárt mint osztályfőnök, tantárgyanként megadhatja, melyik tanár fogja tanítani. 

Végül is, ő kezeli az osztályok felhasználókból való felépítését is.

Admin felhasználóból egy iskolában csak egyre van szükség, de persze ha úgy érzi, hogy ha többen lennének, a jogosultság változtatási jogával, más felhasználókat is feljogosíthat, adminná tehet.

Értelem szerűen az adatok módosítására is lehetőséget kell adjunk, persze a hivatalos adatokat csak ő módosíthatja, más felhasználók csak ami számukra fontos hogy sajátos legyen, például a jelszava, vagy belépési felhasználóneve stb.. módosíthatják.
%%%%%%%%%%%%%%%%%%%%%%%%%%%%%%%%%%%%%

\subsection{Diák}\label{subsec:RESZLETESLEIRAS:cel:diak}
A {\bf Diák} felhasználónk az a felhasználó, aki jegyeket kap, üzenteket és értesítéseket olvas.

Neki joga van látni az összes tantárgya összes jegyét, amelyek mindig az átlagot kiszámítják, esetleg megsúgják, hányas jegyet/jegyeket kéne kapjon, a javításhoz, stb.

Látni fogja a hiányzásainak számát is, lebontva különböző tantárgyakra, a program figyelmeztetni fogja őt, ha elég közel van a kiszabott maximum hiányzások számához.

A tantárgyak tanárait is láthatja, persze csak a nevüket, illetve üzenetet is írhat nekik.

Szülők számára való üzenet küldése teljesen értelmetlen, a saját szüleivel tud beszélni a diák, más szüleivel nincs oka beszélgetni.

Diáktársaival viszont üzenetezhet nyugodtan, és a titkárságra is írhat, ha valami olyan dologról van szó, amivel az osztályfőnöke/tanára is a titkárságra menne. 

Így tehát a Diák felhasználó 3 fajta üzenetet küldhet: Más diáknak, Tanároknak, illetve Titkárságra.

%%%%%%%%%%%%%%%%%%%%%%%%%%%%%%%%%%%%%

\subsection{Tanár}\label{subsec:RESZLETESLEIRAS:cel:tanar}
A {\bf Tanár} felhasználónk az a felhasználó, aki tantárgya(ka)t tanít, jegyeket, hiányzásokat ad illetve osztályt is vezethet, mint osztályfőnök.

\subsubsection{A jegy}\label{subsubsec:RESZLETESLEIRAS:cel:tanar:jegy}
A Tanár egyszerűen jegyeket adhat a diákoknak, ebből kifolyólag láthatja az általa tanított tantárgyakból adott jegyeket diákonként, de másat nem.

Amikor jegyet ad, a Tanár kiválaszthatja, hogy szeretné-e ha a diák szülei mindenképp megnéznék a jegyet, ami azt eredményezi, hogy a szülő valamilyen módon értesülni fog arról, hogy gyermeke, a Diák jegyet kapott.

Viszont ha osztálya van a Tanárnak, mivel a lezárásokat ő kell "intézze"(igazából ez automatikusan megy, csak ő kell elfogadja az eredményeket), az osztályában szereplő összes diák jegyét látnia kell. 

A hiányzásokkal is nagyjából így és ezen okok miatt kell tudjon foglalkozni
%%%%%%%%%%%%%%%%%%%%%%%%%%%%%%%%%%%%%%
\subsubsection{Üzenet}\label{subsubsec:RESZLETESLEIRAS:cel:tanar:uzenet}
A Tanár üzenetet küldhet igazából bárkinek:
\renewcommand{\baselinestretch}{0.98}\normalsize
\begin{description}
    \setlength{\itemsep}{0.04mm}
    \item[Az Adminnak] -- Ha valamely diákjának gondja akadna a felhasználójával, kell tudja értesíteni az admint, hogy valami nincs rendben
    \item[Bármely Szülővel] -- Egy tanár az osztálya bármely diákjának szülőjével kell tudja tartani a kapcsolatot, ez alap dolog, de más diák szüleivel is érdemes kapcsolatot létesíteni, például ha a diák rossz fát tesz a tűzre stb.
    \item[Bármely Diákkal] -- Fontos, hogy ne csak a saját diákjai számára tudjon üzenetet küldeni, mivel más diákokkal is tarthatja a kapcsolatot, teszem fel plusz órát tart, és megbeszélni a részletesebb információkat.
    \item[Titkárral] -- A titkársággal is kell tudjon beszélni, magától étetődő okok miatt.
\end{description}
%%%%%%%%%%%%%%%%%%%%%%%%%%
%%%%%%%%%%%%%%%%%%%%%%%%%%%%%%%%%%%%%

\newpage
\subsection{Szülő}\label{subsec:RESZLETESLEIRAS:cel:szulo}
A {\bf Szülő} felhasználónk az a felhasználó, aki hozzá van rendelve egy, vagy akár több Diákhoz is.

Fontos megemlíteni, hogy a dákok számára a szabadság jog megvan, tehát nem tudja a szülő kilistázni gyermeke jegyeit, csak ha a tanár úgy szeretné, hogy értesüljön gyermeke EGY kapott jegyéről, akkor kap egy értesítést.

Természetesen év végén szintén értesítésben megérkeznek a Diák(ok) lezárásai is, tantárgyanként, illetve összesítve is.

Üzenetezni kifejezetten csak a Tanárokkal, illetve a Titkársággal tud, a többi felhasználó fele nem kell irányuljon ilyen tevékenység.
%%%%%%%%%%%%%%%%%%%%%%%%%%%%%%%%%%%%%

\subsection{Titkár}\label{subsec:RESZLETESLEIRAS:cel:titkar}
A {\bf Titkár} felhasználónk az a felhasználó lényegében a hivatali dolgok lebonyolításához kellenek.

Ha bármi fontos hivatali dolog van, az iskola tagjainak értesülniük kell róla, ezt a titkár fogja megtenni, ugyanakkor ha csak kiemelt személyek érintettek egyes dologban őket külön kell tudják értesíteni.

Tehát a közös értesítésen kívül a Titkárnak jogában áll a felvett tagok közül bárki számára üzenetet küldeni.
%%%%%%%%%%%%%%%%%%%%%%%%%%%%%%%%%%%%%