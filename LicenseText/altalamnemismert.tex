\chapter{Általam még nem ismert, de alkalmazni kívánt technológiák}\label{ch:ALTALAMNEMISMERT}
\begin{osszefoglal}
az előző fejezetben láthattuk, hogy az államvizsgámmal kapcsolatban tudom azt, hogy a front-end része React.js segítségével lesz megvalósítva, a kommunikációt a front-end és back-end között JSON objektumokkal oldom meg, illetve, hogy az adatokat a Wamp segítségével (kezdetben legalábbis) lokális szerveren, mySQL adatbázis segítségével fogom tárolni.

Ebben a fejezetben sor kerül azon technológiák bemutatására, amelyeket szeretnék felhasználni, de még soha nem használtam, csak hallottam/olvastam róluk.
\end{osszefoglal}

\section{Spring}\label{sec:ALTALAMNEMISMERT:spring}
\begin{figure}[h]
	\centering
	\includegraphics[scale=1]{spring}
\end{figure}
A Spring egy nyílt forráskódú, inversion of controlt (IoC) megvalósító Java alkalmazás keretrendszer, melynek elsődleges célja, hogy a vállalati környezetbe szánt Java alkalmazások fejlesztését egyszerűbbé tegye. Igazából a korabeli Java EE szabvány könnyebb alternatívája ként hozták létre, persze azóta a Java EE is változott, sok minden leegyszerűsödött benne, és persze a Spring is kibővült Java EE-ös dolgokkal. Mára már metszetük is létezik.

A Springet a Java eszközök svájci bicskájaként is szokták emlegetni, annyi nem feltétlenül összetartozó, de integrálható eszközt tartalmaz.

Mivel sokat hallottam róla, és sokan ajánlják egy webapplikáció back-endjének megvalósításhoz, és ezis mint a React, egy gyorsan terjedő software keretrendszer, meg szeretném ismerni, és alkalmazni, tapasztalatszerzés, és ismereteim bővítése érdekében.
%%%%%%%%%%%%%%%%%%%%%%%%%%%%%%%%%%%%%%%%%%%%%%%%%%%%%%%%

\section{Felhő}\label{sec:ALTALAMNEMISMERT:felho}

Napjainkban az igazán komoly felhők már nagyon magas szintű szolgáltatásokat nyújtanak a fejlesztők számára. 
Olyanokat, amik a hagyományos asztali környezetben is elérhetőek, például adatbázis kezelés, adattárolás, alkalmazásszerver futtatás, de olyan magas fokú adatbiztonságot és rendelkezésre állást tudnak biztosítani, amikről eddig még álmodni sem lehetett.

Ez így nagyon jól hangzik, és ha jól haladok a dolgozattal, a felhőt is mint új technológiát valamilyen szinten szeretném alkalmazni.

\begin{figure}[h]
	\centering 
	\begin{tabular}{ccc}
		\includegraphics[scale=0.2]{Azure}
		&
		\includegraphics[scale=0.7]{heroku}
	\end{tabular}
\end{figure}
%%%%%%%%%%%%%%%%%%%%%%%%%%%%%%%%%%%%%%%%%%%%%%%%%

